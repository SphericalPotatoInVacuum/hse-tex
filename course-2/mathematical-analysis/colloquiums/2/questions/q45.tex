\subsection{Дайте определение цилиндрических координат (формулы, область задания, координатные линии, матрица Якоби перехода, якобиан)}

Цилиндрические координаты $(r, \varphi, z)$ в пространстве $(x, y, z)$ вводятся формулами
\[ x = r \cos \varphi \]
\[ y = r \sin \varphi \]
\[ z = z \]
При этом $U = (0; +\infty) \times [0; 2\pi) \times \RR$ и $X = \RR^3 \setminus \{(0, 0, z) | z \in \RR \}$
Выколотая ось $z$ при этом называется полярной осью. Угол $\varphi$ называется азимутом или азимутальным углом.

Координатные линии $r$ -- лучи, выходящие из точки на полярной оси перпендикулярно полярной оси. Координатные линии $\varphi$ --
окружности с центром на полярной оси, расположенные в плоскостях, перпендикулярных полярной оси. Координатные линии $z$ -- прямые,
параллельные полярной оси.

Матрица Якоби перехода имеет вид:

$
\begin{pmatrix}
    \cos \varphi & -r \sin \varphi & 0\\
    \sin \varphi & r \cos \varphi  & 0\\
    0            & 0               & 1
\end{pmatrix}
$

Якобиан равен $r$.